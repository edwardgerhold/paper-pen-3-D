\documentclass{letter}
\begin{document}
\begin{center}
\Large{\texbf{Three dimensional coordinates into two dimensional coordinates conversion}}
\end{center}

Written by Edward Gerhold

Version 0.1 (draft paper)

Definitions

[Picture of a 3-D coordinate system with ijk-vectors on the axes pointing
into three Directions]

Let $\phi_n$ be the set of axis angles, one for each axis. The angles start
at the same place, at the number zero, you have to arrange the $x$, $y$, and
$z$ axes like on a piece of paper around the unit circle by giving them the
appropriate angles. All three angles start at the default at zero.

$\phi_n := { \phi_x, \phi_y, \phi_z }$

Let $e_n$ be the set of three two dimensional unit base vectors, namely ex,
$e_y$ and $e_z$, they point on the two dimensional plane into three directions
and represent the axes of the three dimensional coordinate system.

$e_n := { e_x, e_y, e_z }$
 
To guess no numbers, it´s easier for us, to go around the unit circle by
the angles of the unit vectors, and to use cosine and sine for the correct
$x$-distance and $y$-distance. For help, you should remember this parametrization
of $x$ and $y$ from the unit circle.

$x = r \cos \phi$
$y = r \sin \phi$

Modeling the three two dimensional base vectors with this information,
we get the following three two dimensional base vectors. 


$e_x := (r_x\cos(\phi_x), r_x\sin(\phi_x) )^T$
$e_y := (r_y\cos(\phi_y), r_y\sin(\phi_y) )^T$
$e_z := (r_z\cos(\phi_z), r_z\sin(\phi_z) )^T$


Remark. One for each component of $(x,y,z)$ By multiplying with, we move the 
points into their directions for the unit of the $(x,y,z)$ components.
The values of $r_x, r_y$ and $r_z$ decide, how long one unit into each
direction is. To preserve affine graphical transformations all three
axes should have the same unit length, which can generally be enlarged
or made smaller than unit length. By default the resulting vector of the cos 
and sin Terms has unit length, if you don´t multiply with $r_x, r_y$ and $r_z$. 

The other help we take is from the orthogonal base formula.
The sum of the basis multiplied with the coordinates is nothing
new. But literature explains only how to multiply square matrices
or coordinates and bases with equal dimensions.

$x' = \displaystyle\sum^{n} e_nx_n$

To make it short, each $(x,y,z)$ coordinate has to be multiplied for the new $(x',y')$
with it´s corresponding term of the unit vectors in the matrix. That means,
to sum the products with $(x,y,z)$ and the cos terms up for $x'$ and to sum the products
of $(x,y,z)$ and the sin terms up for $y'$. This is the same as imagining walkin left and
right with cos and up and down with sine. Or mathematically adding positive or negative values.


$x' = xr_x\cos(\phi_x) + yr_y\cos(\phi_y) + zr_z\cos(\phi_z)$
$y' = xr_x\sin(\phi_x) + yr_y\sin(\phi_y) + zr_z\sin(\phi_z)$


Let A be the matrix containing the three unit vectors in order, one each
column. You get a 2x3 matrix, which i call the Gerhold Matrix to distinguish 
it from other matrices.

$A := (e_x, e_y, e_z)$
$  = (r_x\cos(\phi_x), r_y\cos(\phi_y), r_z\cos(\phi_z);$
$     r_x\sin(\phi_x), r_y\sin(\phi_y), r_z\sin(\phi_z))$


Theorem (Fundamental Theorem of converting 3-D Points into 2-D Points):


If you multiply the matrix containing the three two-dimensional unit vectors
with the three coordinate points $(x,y,z)$, the result is a two coordinate point, 
$(x',y')$. This point $(x',y')$ is the correct point on the two dimensional plane,
representing the point from the three dimensional coordinate system we display.


$A(x,y,z) = (x',y')$

in other words

$x' = xr_x\cos(\phi_x) + yr_y\cos(\phi_y) + zr_z\cos(\phi_z)$
$y' = xr_x\sin(\phi_x) + yr_y\sin(\phi_y) + zr_z\sin(\phi_z)$

Proof:
[example calculation or higher math variable proof]

$x' = xr_x\cos(\phi_x) + yr_y\cos(\phi_y) + zr_z\cos(\phi_z)$
$y' = xr_x\sin(\phi_x) + yr_y\sin(\phi_y) + zr_z\sin(\phi_z)$

Corollary (Converting any Dimensions down to less dimensions)

The theorem can be extended to more dimensions, for example can four two-dimensional
vectors represent a 4-D space on the 2-D plane. They get converted into the correct
2-D points. For Example, if you use a 2x4 matrix and convert all points at each 
instance of t you have a moving object into the direction of the fourth vector. 

$A := (e_x, e_y, e_z, e_t)$
$  = (r_x\cos(\phi_x), r_y\cos(\phi_y), r_z\cos(\phi_z), r_t\cos(\phi_t),$
$     r_x\sin(\phi_x), r_y\sin(\phi_y), r_z\sin(\phi_z), r_t\sin(\phi_t))$

$A(x,y,z,t)=(x',y')$

The same method can be used to convert any other number of dimensions to the $xy$-plane.
If you know the base vectors for the other dimensions you can convert them as well.
\end{document}
