Three dimensional coordinates into two dimensional coordinates conversion

Written by Edward Gerhold
http://linux-swt.de
email:edward.gerhold@googlemail.com

Definitions

[Picture of a 3-D coordinate system with ijk-vectors on the axes pointing
into three directions]

Let phin be the set of axis angles, one for each axis. The angles start
at the same place, at the number zero, you have to arrange the x, y, and
z axes like on a piece of paper around the unit circle by giving them the
appropriate angles.

phin = { phix, phiy, phiz }

Let en be the set of three two dimensional unit base vectors, namely ex,
ey and ez, they point on the two dimensional plane into three directions
and represent the axes of the three dimensional coordinate system.

en = { ex, ey, ez }

To guess no numbers, it´s easier for us, to go around the unit circle by
the angles of the unit vectors, and to use cosine and sine for the correct
x-distance and y-distance. For help, you should remember this parametrization
of x and y from the unit circle.

x = r cos phi
y = r sin phi

Modeling the three two dimensional base vectors after this we get the following

ex = (r*cos(phix), r*sin(phix) )T
ey = (r*cos(phiy), r*sin(phiy) )T
ez = (r*cos(phiz), r*sin(phiz) )T

To make it short, each x,y,z coordinate has to be multiplied for the new x' and
the new y' coordinate with it´s correspondin term of the unit vector. That means,
to summarize the cos terms for x' and to summarize the sin terms for y'.

x' = x*r*cos(phix) + y*r*cos(phiy) + z*r*cos(phiz)
y' = x*r*sin(phix) + y*r*sin(phiy) + z*r*sin(phiz)

Let A be the matrix containing the three unit vectors in order, one each
column. You get a 2x3 matrix, which i call the Gerhold Matrix to distinguish 
it from other matrices. 

A = (ex, ey, ez) 
  = (r*cos(phix), r*cos(phiy), r*cos(phiz);
     r*sin(phix), r*sin(phiy), r*sin(phiz))

Theorem (Fundamental Theorem of converting 3-D Points into 2-D Points)

If you multiply the matrix containing the three two-dimensional unit vectors
with the three coordinate points (x,y,z), the result is a two coordinate point,
which is the correct point on the two dimensional plane representing the point
in the three dimensional coordinate system we display.

A(x,y,z) = (x',y')

Corollary (Converting more Dimensions)

The theorem can be extended to more dimensions, for example can four two-dimensional
vectors represent a 4-D space on the 2-D plane. They get converted into the correct
2-D points. For Example, if you use a 2x4 matrix and convert all points at each 
instance of t you have a moving object into the direction of the fourth vector. 
A(x,y,z,t)=(x',y')
