 \documentclass[a4paper]{article}
\usepackage[hmargin=1in, vmargin=1in]{geometry}
\usepackage{makeidx}
\usepackage{fancyhdr}
\pagestyle{fancy}
\usepackage[pdftex]{graphicx}
\usepackage{amsmath}
\usepackage{amssymb}
\usepackage{listings}
\usepackage{natbib}
%\usepackage{etex}
%\usepackage{m-pictex}
\makeindex
\begin{document}
\begin{center}
\title{Three dimensional coordinates into two dimensional coordinates transformation}\\
\author{Edward Gerhold}
Three dimensional coordinates into two dimensional coordinates transformation.\\
Overworked english edition. Almost completly rewritten chapters from top to bottom.
\date{\today}
\maketitle

Version 0.4.0-under-construction\\
old file: gerhold-3dto2d-0.3.99-with-mistakes.pdf\\
german draft: gerhold-3dto2d-de.pdf\\

\textbf{Remark. This is a development version. With proper mistakes. The coordinate system itself is easy, useful and a working tool.}

\end{center} 

\tableofcontents\\

\section{Introduction}

\section{Designing a 2x3 coordinate system}
\subsection{Indizes and notations in this document}
\subsection{Axis angles}
\subsection{Axis units}
\subsection{Axis vectors}
\subsection{Time to show the operation - sums of three scaled 2-D position vectors.}
\subsection{Some topology}

\section{Transformation tools}
\subsection{In place of a 3x3 basis}
\subsection{The functional f(x)}
\subsubsection{Natural $\left<f,x\right>$}
\subsubsection{Examples for $f \circ g$}
\subsubsection{Derivative}
\subsubsection{Integral of the derivative}
\subsubsection{Integrals of f(x)}

\subsection{The matrix}
\subsubsection{Perfect 2x3 coordinate system}
\subsubsection{Examples of 2x3 coordinate systems}
\subsubsection{$\nabla f ^{T}$ Jacobi matrix of f(x)}
\subsubsection{Singular value decomposition} 
\subsection{Projecting z onto a vector}

\section{Transformation behaviour}
\subsection{Linear transformation}
\subsection{Linear map, bounds on the operator}
\subsection{Sequences and Convergence}

\section{Computer Implementation}
\subsection{Generic Computer Code}
\subsection{EcmaScript 6 Example Code}

\section{Basis vs. Linear Combination}
\subsection{Orthogonality}
\subsubsection{Orthogonal Projection}
\subsection{Gram-Schmidt applied onto 2x3}

\section{Corollary}
\subsection{Transforming four dimensions with 2x4}
\subsection{3-D curve of a complex sequence, family surfaces}
\subsection{My plane in front of the drawn side of the space}
\subsection{ker(A), f=0 points are on the line into and through the origin $\perp$ to the planes origin}
\subsection{Covered points on a line normal to the front of the drawn}

\section{Summary}

\appendix

\section{Vector spaces}
\subsection{Basic tools}

\section{To be continued}
\subsection{Measuring projected objects}


\end{document}